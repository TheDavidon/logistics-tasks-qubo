\documentclass{article}

\usepackage[utf8x]{inputenc}
\usepackage[russian]{babel}
\usepackage{amsmath}

\title{QUBO формулировка для задачи с одним агентом и типом товара}
\date{}

\begin{document}

\maketitle

\section{Общая логика модели}
Задача, сформулированная в общем виде, переводится в модель QUBO. Первый шаг - частный случай с одним агентом и одним видом товара. \\
\\
Основные переменные содержат всю информацию об итоговом пути и действиях агента. Вспомогательные переменные нужны для построения гамильтонианов и проверки корректности полученного решения. Константы - величины, известные из входных данных. Вспомогательные выражени - перевод бинарной кодировки величин в их численное значение \\
\\
Изначально дается неориентированный связный взвешенный (без отрицательных весов) граф. По нему строится полный граф расстояний между вершинами, а также запоминаются все кратчайшие пути между парами вершин (алгоритм Флойда с памятью) для последующего восстановления пути агента.\\
\\
Гамильтониан, который преобразуется в квадратичную форму для QUBO, разделен на несколько слагаемых. Гамильтонианы корректности системы отвечают за то, что по бинарным переменным можно восстановить корректный путь и действие:
\begin{itemize}
    \item правильная начальная и конечная вершина
    \item за каждый ход посещена единственная вершина
    \item за каждый ход совершено одно действие, при этом оно совершено на посещенной в этот ход вершине
    \item после каждого хода величина груза агента не меньше нуля и не превышает его вместимость
    \item после последнего хода величина груза агента равна нулю
    \item за весь путь из каждого склада взято груза не больше, чем в нем изначально находилось
    
\end{itemize} \\
\\
Гамильтонианы удовлетворения потребностей отвечают за то, что в конце пути в каждой вершине потребителе величина отданного агентом груза равна величине потребности вершины. Гамильтонианы целевой функции отвечают за минимальность суммарной длины пути агента.\\
\\
В итоговой сумме каждый гамильтониан входит в коэффициентом, который нужен, чтобы каждое ограничение выполнялось и не вытесняло остальные. Их подбор (будет) приведен ниже.




\section{Переменные и вспомогательные величины}


\subsection{Основные бинарные переменные}
$x_{ij}$ - на i-ом шаге была посещена j-ая вершина. Кол-во переменных: \(qn\)\\
$y_{ijk}$ - k-ая переменная в кодировке величины действия на j-ой вершине на i-ом ходу. Кол-во переменных: \(qn(M + 1)\)\\

\subsection{Вспомогательные бинарные переменные}
$z_{ij}$ - j-ая переменная в кодировке величины груза агента на i-ом ходу. Кол-во переменных: \(q(M+1)\)\\
$w_{ij}$ - j-ая переменная в кодировке величины суммарного действия на i-ой вершине. Кол-во переменных: \(U\) \\

Суммарное кол-во переменных: \(q(nM + 2n + M + 1)+ U\)


\subsection{Константы}
\[n - \text{кол-во вершин}\]
\[T_{j} = \left\{
  \begin{aligned}
    & 1, \text{ если j - склад}
    \\
    & -1, \text{ если j - потребитель}
    \end{aligned}
\right.\]
\[R_{j} - \text{потребности j-ой вершины, если j - потребитель, иначе размер склада}\]
\[C - \text{вместимость агента}\]
\[s, f - \text{индексы начальной и конечной вершины пути агента}\]
\[d_{ij} - \text{длина кратчайшего пути между вершинами i и j}\]
\[q - \text{кол-во вершин в пути агента}\]
\[M = \left \lfloor{log_{2} C}\right \rfloor \]
\[U_{i} = \left \lfloor{log_{2} R_{i}}\right \rfloor \]
\[U = \sum_{i=0}^{n-1}U_{i}\]






\subsection{Вспомогательные выражения}

\[B_{ij} = \sum_{k=0}^{M-1} 2^ky_{ijk} + y_{ij M} (C + 1 - 2^M)\]

\[L_{i} = \sum_{j=0}^{M-1} 2^j z_{ij} + z_{i M} (C + 1 - 2^M)\]

\[S_{i} = \sum_{j=0}^{U_{i}-1} 2^j w_{ij} + w_{i U_{i}} (R_{i} + 1 - 2^{U_{i}})\]



\section{Гамильтонианы}

\subsection{Гамильтонианы корректности системы}

\[H_{correct \ path} \ = \sum_{i=0}^{q - 1} (1-\sum_{j=0}^{n-1}x_{ij})^2\]\\
\[H_{correct \ end \ points} \ = \ (2 - x_{0,s} -x_{q-1, f})^2 \]\\
\[H_{correct \ action, \ i} \ =\ \sum_{j=0}^{n-1} B_{ij} (1 - x_{ij}) \]\\
\[H_{correct \ actions} \ =\ \sum_{i=0}^{q-1} H_{correct \ action, i} \]\\
\[H_{correct \ load, \ i} \ =\ (L_{i} - \sum_{k=0}^{i}\sum_{j=0}^{n-1} B_{kj} T_{j})^2 \quad, \ i\le q-2\]\\
\[H_{correct \ load, \ q-1} \ =\ (\sum_{k=0}^{q-1}\sum_{j=0}^{n-1} B_{kj} T_{j})^2\]\\
\[H_{correct \ loads} \ =\ \sum_{i=0}^{q-1}H_{correct \ load, \ i}\]\\
\[H_{correct \ storage \ action, \ j } \ =\ (S_{j} - \sum_{i=0}^{q-1} B_{ij} )^2, \quad \text{j - индекс вершины склада}\]\\
\[H_{correct \ storage \ actions} \ =\ \sum_{j \in storage \ vertices} H_{correct \ storage \ action, \ j}\]\\


\subsection{Гамильтонианы удовлетворения потребностей}
\[H_{satisfied \ consumer, \ j } \ =\ (R_{j} - \sum_{i=0}^{q-1} B_{ij} )^2, \text{  j - индекс вершины потребителя}\]\\
\[H_{satisfied \ consumers} \ =\ \sum_{j \in consumer \ vertices} H_{satisfied \ consumer, \ j }\]\\

\subsection{Гамильтонианы целевой функции}
\[H_{path \ length} \ =\ \sum_{u, v = 0}^{n-1} \sum_{i=0}^{q-2}x_{i,u}x_{i+1, v} d_{uv}\]\\

\section{Подбор множителей для итоговой суммы гамильтонианов}
TODO


\end{document}